\chapter{Ranking Model}

Consider the following model:
all of the weights of edges are independently sampled from the same distribution $D$ which is unknown to us.
The concept ``competitive ratio $\alpha$'' in this particular model means that for any instance $G$ and distribution $D$, by taking the expectation over all the possible weights and all the permutations, we have $\frac{E[ALG]}{E[OPT]} \ge \alpha$.

\begin{theorem}\label{rothm}
    Simultaneous stopping rule achieves a constant competitive ratio.
\end{theorem}

\begin{proof}

    WLOG we may assume that applicants comes according to the order $\{v_1, v_2, \dots, v_n\}$ since there is no difference between them.

    Firstly let's introduce some notations:

    \begin{itemize}
        \item Event $B(u_i, v_j)$ means that applicant $v_j$
            values most to firm $u_i$.
        \item Event $P(u_i, v_j)$ means that firm $u_i$ sends
            an offer to $v_j$.
        \item Event $A(u_i, v_j)$ means that firm $u_i$ sends
            an offer to $v_j$ and $v_j$ accepts it.
    \end{itemize}

    Fix a particular firm $u \in U$,
    first bound the probability that it will get the best applicant
    according to its preference.
    \begin{align*}
        \Pr(u\text{ gets the best}) = &\sum_{i=1}^{n} \Pr(B(u, v_i)) \times \Pr(A(u, v_i)|B(u, v_i)) \\
                                = &\sum_{i=r}^{n} \frac{1}{n} \times \Pr(P(u, v_i)|B(u, v_i)) \\
                        & \times \Pr(A(u, v_i)|B(u, v_i),P(u, v_i))
    \end{align*}
    
    Given $v_i$ is the best applicant for $u$,
    once we ensure that the previous $(i-r)$ applicants are
    no better than the threshold, $u$ will be free when $v_i$ comes and will
    send her an offer.
    That is, we need to ensure that over the last $(i-1)$ applicants,
    the best is among the first $(r-1)$ ones.
    Since all the applicants arrive in a random order, this probability would be $\frac{r-1}{i-1}$.
    Thus

    $$\Pr(P(u, v_i) | B(u, v_i)) \ge \frac{r-1}{i-1}$$

    %$$\Pr(u\text{ gets the best}) \ge \sum_{i=r}^{n} \frac{1}{n} \times \frac{r-1}{i-1} \times \Pr(A(u, v_i)|B(u, v_i),P(u, v_i))$$


    If $v_i$ receives only one offer which is from $u$, $v_i$ can only choose $u$.
    And the condition holds when the value of $v_i$
    for each $u' \neq u$ does not exceed the threshold set by $u'$.
    In other words, for each $u' \neq u$, over the first $r-1$ applicants together with $v_i$,
    the best applicant for $u'$ is not $v_i$.
    These $m-1$ events (each with probability $\frac{r-1}{r}$) are independent from each other,
    and are all independent from $B(u, v_i)$ and $P(u, v_i)$.
    Thus
    $$\Pr(A(u, v_i)|B(u, v_i), P(u, v_i)) \ge (\frac{r-1}{r})^{m-1}$$

    To sum up:
    $$\Pr(u\text{ gets the best}) \ge \sum_{i=r}^{n} \frac{1}{n} \times \frac{r-1}{i-1} \times (\frac{r-1}{r})^{m-1}$$

    Let $p = \frac{r}{n}$ be a constant, and assume that $m \le \alpha n$ where $\alpha \in (0,1]$ is a parameter.

    Then $(\frac{r-1}{r})^{m-1} \ge (1-\frac{1}{r})^{\alpha n} = ((1-\frac{1}{r})^{r})^{\frac{\alpha}{p}}$ and

    $$\Pr(u \text{ gets the best}) \ge ((1 - \frac{1}{r})^r)^{\frac{\alpha}{p}} \sum_{i=r}^{n} \frac{1}{n}\frac{r-1}{i-1}$$

    Now let $n$ goes to infinity and it approaches the integral:

    $$f(p) = e^{-\frac{\alpha}{p}} \int_{p}^{1} \frac{p}{x} dx = -p \ln(p) e^{-\frac{\alpha}{p}}$$
    which is a constant depending on the choice of $p$.
    Thus, for each firm $u$, it
    has at least a constant probability $f(p)$ to get its best applicant, then we have
    \begin{align*}
        E[\text{value of the applicant }u\text{ gets in }ALG]
        & \ge f(p) \times E[\text{value of the best applicant for }u] \\
        & \ge f(p) \times E[\text{value of the applicant }u\text{ gets in }OPT]
    \end{align*}
    
    Therefore, it is clear to see that
    \[ \frac{E[ALG]}{E[OPT]} \ge f(p), \]
    and we achieves a constant competitive ratio.
\end{proof}

Note that, in the analysis above, it is not necessary that all the weights of edges are independent and identically distributed.
All we need is that firms' preference lists are independent from each other. 
That is, the rank of an applicant $v$ for a firm $u_i$ has nothing to do with the rank of $v$ for another firm $u_j$.

To generalize the random rank model above, we weaken the restriction by
adding correlations between the weights of edges incident to the same applicant,
and introduce a new model as follows.

