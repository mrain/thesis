\chapter{General Model}

\section{Formal Definition}

First of all, we need to formally define the objective function in our problem.
Generally speaking, for a given instance $G = (U, V, w)$, considering every permutation of the applicants,
we have the value of optimal assignment (denoted by $OPT$), and the value of the algorithm's outcome (denoted by $ALG$).
By saying ``competitive ratio $\alpha$'' in the general case, we mean that for any instance $G$, $\frac{E[ALG]}{E[OPT]} \ge \alpha$.
In the following sections, we will present more specific settings, and will define the objectives accordingly.

\section{Simultaneous stopping rule}

Then we turn to the algorithms.
For the applicants, everyone adopts the same simple strategy:
select the most valuable offer among those she have got.
And for firms, we propose two algorithms based on classical online
secretary problem's approaches.

One direct algorithm is \textbf{simultaneous stopping rule}:

\begin{enumerate}
    \item Each firm $u$ observes and rejects the first $(r - 1)$ applicants.
        Denote the value of the best applicant among them by $t_u$ which
        is the threshold. Typically we set $r$ to be a constant fraction of $n$
        such as $\lfloor \frac{n}{e} \rfloor + 1$.

    \item For each applicant coming in later, each firm $u$ who hasn't been
        matched before sends her an offer if and only if
        its value exceeds $t_u$.
\end{enumerate}
%reject first $r - 1$ applicants and use their value to
%set a threshold, and then propose everyone whose value exceeds
%that threshold as long as it is still available.

%Typically we set $r = \lfloor \frac{n}{e} \rfloor + 1$.
But this algorithm does not work very well in general case if
we let $r = \lfloor \beta n \rfloor + 1$ for some constant $\beta < 1$.

\begin{example}\label{example1}
    Let $m = \Theta(n)$. There are two kinds of applicants.
    First we have $a = \Theta(\log(n))$ `good' applicants with weight $2 + \epsilon_{u,v}$
    for all edges incident to them.
    The rest applicants are marked as `bad' with weight $1 + \epsilon_{u,v}$
    for all edges incident to them
    ( $\abs{\epsilon_{u,v}} < \frac{1}{2}$ is only used to avoid ties).
\end{example}

Note that all $\epsilon_{u,v}$ could be arbitrarily small. So
in the following discussion, we can ignore them.
For any permutation of the applicants, $OPT = m + a$, thus $E[OPT] = m + a$.
Now it's time to give a upper bound for $E[ALG]$.

In the first $(r - 1)$ rounds, the probability that no firm sees
a `good' applicant is:

$$\prod_{i=0}^{\lfloor \beta n \rfloor} \frac{n - a - i}{n - i}
\le (1 - \frac{a}{n})^{\lfloor \beta n \rfloor} \le (1 - \frac{a}{n})^{\beta n - 1}$$

And if they did observe a `good' applicant, no `bad' applicant can be
matched since threshold for each firm is set to be near $2$.

Therefore
\begin{align*}
    E[ALG]  & \le (1 - \frac{a}{n})^{\beta n - 1} \times (m + a)
                + (1 - \prod_{i=0}^{\lfloor \beta n \rfloor} \frac{n - a - i}{n - i}) \times 2a \\
            & \le \frac{n}{n - a} e^{- \beta a} \times (m + a) + 2a,\\
    \frac{E[ALG]}{E[OPT]} & \le \frac{n}{n - a} e^{- \beta a} + \frac{2a}{m+a}
            = \frac{1}{n^{\Theta(1)}} + \Theta(\frac{\log n}{n}).
\end{align*}

\begin{corollary}
In the general case, simultaneous stopping rule cannot get better competitive ratio than $\Theta(\frac{\log n}{n})$
if we set $r = \lfloor \beta n \rfloor + 1$ for some constant $\beta < 1$.
\end{corollary}

The problem encountered here is that, without global information, firms
are competing over a small set of applicants and left all the others aside.


\section{Simultaneous stopping rule with $m$ slots}

To tackle this problem, we slightly modify the algorithm above.
Instead of only one threshold, we set $m$ thresholds.
Like the algorithm for multiple choice secretary problem proposed
in \emph{TODO}, we call it \textbf{simultaneous stopping rule with m
thresholds}:

\begin{enumerate}
    \item Each firm $u$ observes and rejects the first $(r - 1)$ applicants.
        Choose the best $m$ applicants to form a threshold set $T_u$.
        Denote their values by $t_{u,1}, t_{u,2}, \dots, t_{u,m}$
        in decreasing order.
        (Some of them are set to be $-\infty$ if not enough).

   \item For each applicant $v$ arriving later and each firm $u$, if the
       applicant $v$'s value $w(u,v)$ exceed $t_{u,m}$,
       then add it to $T_u$ and the least valuable one in $T_u$ is removed.
       That is, firm $u$ always keeps the best $m$ values in $T_u$.
       And $v$ will get an offer from $u$ if and only if $w(u,v)$ is added to $T_u$ and
       the one removed from $T_u$ is $-\infty$ or has an arrival time less than $r$.


\end{enumerate}



\emph{TODO: more comments}

Later on, we are going to propose more specific models to see how well the two algorithms could work.
In Section \ref{} and \ref{}, simultaneous topping rule can achieve constant competitive ratio under certain conditions;
while in Section \ref{}, the algorithm with m thresholds works well.


