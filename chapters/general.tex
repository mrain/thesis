\chapter{General Model}\label{chap:general}

\section{Formal Definition}

Before giving our results, we need to first define the model
of our problem and its objective function.
Generally speaking, instead of one, we have $n$ firms and each of them
wants to hire one secretary. There are $m$ applicants waiting outside
for their interview. The applicants will come in one by one and all
$n$ firms would interview her simultaneously. After one interview each
of the firms would grade the applicant who has just been interviewed
a score, and then decide whether to accept her or not. If the applicant
receives offers, she could choose any one of them.
In the following part we assume that the applicant would always choose the
offer with the highest score, for it probably has the highest salary
in real life.

More formally, in a bipartite matching perspective.
For a given edge weighted complete bipartite $G = (U, V, w)$,
where $U$ corresponds to the set of $m$ firms, 
$V$ corresponds to the set of $n$ applicants
and $w$ is the weight function corresponding to the score firm $u$
graded applicant $v$.
For simplicity we assume that no two edges have the same weights.

The goal is to design a decentralized algorithm for each firm to
(i) maximize the social welfare, i.e.\ find a matching with the 
largest possible sum of scores and (ii) for each firm by adopting the
proposed algorithm they can get their optimal applicants,
i.e\ the best applicant they can get among all possible maximum weighted
bipartite matching. By decentralization it means that, there is no
supervisor doing such an assignment and each firm has
to run the algorithm independently without any communication or
any global information.

We have the value of optimal assignment (denoted by $OPT$), 
and the value of the algorithm's outcome (denoted by $ALG$).
By saying ``competitive ratio $\alpha$'' in the general case, 
we mean that for any instance $G$, $\frac{E[ALG]}{E[OPT]} \ge \alpha$.
In the following sections, we will present more specific settings, 
and will define the objective functions accordingly.

\section{Simultaneous stopping rule}

Then we turn to the algorithms. In the following part we will provides
two simple algorithms, and see how well they can solve this problem in
different scenarios.

It is natural to think of whether we cound simply apply the classical
stopping rule for online secretary problem here.
So here comes our first approach -- \textbf{simultaneous stopping rule}:

\begin{itemize}
    \item 
        The \emph{observation phase}:
        each firm $u$ observes and rejects the first $(r - 1)$ applicants.
        Denote the value of the best applicant among them by $t_u$ which
        is the threshold. 
        Typically we set $r$ to be a constant fraction of $n$
        such as $\lfloor \frac{n}{e} \rfloor + 1$.

    \item 
        The \emph{selection phase}:
        for each applicant who comes in later, 
        each firm $u$ who hasn't been matched before sends her an 
        offer if and only if its value exceeds $t_u$.
\end{itemize}

This algorithm is simple and straightforward. But unfortunately
it does not work very well in general case.
Here is a counterexample:
let $r = \lfloor \beta n \rfloor + 1$ for some constant $\beta \in (0, 1)$.

\begin{example}\label{example1}
    Let $m = \Theta(n)$. There are two kinds of applicants.
    First we have $a = \Theta(\log(n))$ ``good" applicants with 
    weight $2 + \epsilon_{u,v}$
    for all edges incident to them.
    The rest applicants are marked as ``bad" with 
    weight $1 + \epsilon_{u,v}$
    for all edges incident to them
    ( $\abs{\epsilon_{u,v}} < \frac{1}{2}$ is only used to avoid ties
    and they could be arbitrarily small so we can ignore them for
    simplicity of the analysis).
\end{example}

For any permutation of the applicants, it is clear that $OPT = m + a$ by
choosing $a$ ``good" applicants and $m - a$ ``bad" applicants.
To show that simultaneous stopping rule fails to handle this setting,
we are going to give a upperbound for $E[ALG]$.
The upperbound is established by stating that (i) it is almost sure
that the firms can witness a ``good" applicant in the observation phase and
(ii) in this case, only a few firms cound successfully find their
secretaries.

In the first $(r - 1)$ rounds, the probability that no firm observes
a `good' applicant is:

$$\prod_{i=0}^{\lfloor \beta n \rfloor} \frac{n - a - i}{n - i}
\le (1 - \frac{a}{n})^{\lfloor \beta n \rfloor} \le (1 - \frac{a}{n})^{\beta n - 1}$$

We first observes that this function $(1 - \frac{a}{n})^{\beta n}$
is asymptotically $e^{-\beta a}$ as $n$ goes to infinity.
And we set $a = \Theta(\log(n))$, so this probability goes to $0$
as $n$ grows.

The second thing is, 
if the firms did observe a ``good" applicant in the observation phase, 
no firm could give a offer to ``bad" applicants in the selection 
phase according to the protocol
since threshold for each firm is set to be near $2$.

Therefore
\begin{align*}
    E[ALG]  & \le (1 - \frac{a}{n})^{\beta n - 1} \times (m + a)
                + (1 - \prod_{i=0}^{\lfloor \beta n \rfloor} \frac{n - a - i}{n - i}) \times 2a \\
            & \le \frac{n}{n - a} e^{- \beta a} \times (m + a) + 2a,\\
    \frac{E[ALG]}{E[OPT]} & \le \frac{n}{n - a} e^{- \beta a} + \frac{2a}{m+a}
            = \frac{1}{n^{\Theta(1)}} + \Theta(\frac{\log n}{n}).
\end{align*}

\begin{corollary}
In the general case, simultaneous stopping rule cannot get better competitive ratio than $\Theta(\frac{\log n}{n})$
if we set $r = \lfloor \beta n \rfloor + 1$ for some constant 
$\beta \in (0, 1)$.
\end{corollary}

The problem encountered here is that, without global information, firms
will be competing over a small set of elite applicants 
and left all the others aside. While their choices could be more flexible
to avoid this kind of tragedy.


\section{Simultaneous stopping rule with $m$ slots}

To tackle this problem, we slightly modify the algorithm above.
Instead of only one threshold, we set $m$ thresholds.
Like the virtual algorithm for multiple choice secretary problem proposed
in section~\ref{multichoiceproblem}, 
we call it \textbf{simultaneous stopping rule with m thresholds}:

\begin{itemize}
    \item 
        The \emph{observation phase}:
        Each firm $u$ observes and rejects the first $(r - 1)$ applicants,
        then choose the best $m$ applicants among them 
        to form a threshold set $T_u$.
        Denote their scores by $t_{u,1}, t_{u,2}, \dots, t_{u,m}$
        in decreasing order.
        (Some of them are set to be $-\infty$ if not enough).

   \item 
       The \emph{selection phase}:
       For each applicant $v$ who comes in later and each firm $u$, if the
       applicant $v$'s value $w(u,v)$ exceed $t_{u,m}$,
       then add it to $T_u$ and the least valuable one in $T_u$ is removed.
       That is, firm $u$ always keeps the best $m$ values in $T_u$.
       And $v$ will get an offer from $u$ if and only if $w(u,v)$ 
       is added to $T_u$ and the one removed from $T_u$ 
       is either $-\infty$ or has an arrival time less than $r$.
\end{itemize}

Similar you can define a new algorithm based on the optimistic algorithm for
multiple-choice secretary problem, and prove the results with this new
algorithm. But it will be a lot more complicated, so here we do only
focus on the variant of virtual algorithm.

As you can see the choices firms have by adopting simultaneous stopping
rule with $m$ slots are more flexible. They all have at most $m$ chances
to find their desired secretaries.
It will be shown later that this simultaneous stopping rule works well
for example~\ref{example1} above.
But the flexibility brought out more problems.
Here is a simple bad example for simultaneous stopping rule.

\begin{example}\label{example2}
    Let $m = \Theta(n)$. The score that firms grade applicants are all
    $1 + \epsilon_{u,v}$ except for a special firm $u^*$ and a special
    applicant $v^*$. The score $u^*$ gives $v^*$ is denoted by $s$,
    and $s$ can be arbitrarily large.
    About $\epsilon_{u, v}$, the same as in example~\ref{example1}, 
    $\abs{\epsilon_{u, v}}$ could be arbitrarily small.
    One additional requirement is that, for any 
    applicant $v$, $\epsilon_{u^*, v} > 0$ and for any firm $u \neq u^*$, 
    $\epsilon_{u, v} < 0$.
\end{example}

It is observed that, with the additional requirement, once $u^*$ send its
offer, the applicant who receives it would definitely choose $u^*$.
Another observation is that, 
the performance of simultaneous stopping rule is dependent on
the probability that $v^*$ is matched to $u^*$ since the score which
$u^*$ grade $v^*$ can be arbitrarily large.
Then we are going to show that this probability is relatively small.

Assume that $v^*$ comes in for an interview at the $i$-th position where
$i \ge r$. If we want $u^*$ to hold $v^*$, two things have to be ensured:
(i) $u^*$ hasn't sent its offer yet, (ii) $u^*$ has to send its offer to
$v^*$. Clearly that condition (ii) directly follows from condition (i).
As for the condition (i), it is equivalent to the situation that, for $u^*$
the best $m$ applicants who comes in before applicant $v^*$ 
has an arrival time before $r$.
And because the incoming order is uniformly at random,
this happens with a probability of
$$\frac{r - 1}{i - 1} \frac{r - 2}{i - 2} \cdots \frac{r - m}{i - m}
\le \left(\frac{r - 1}{i - 1}\right)^m$$
Therefore the probability of $u^*$ getting $v^*$ is no greater than
$$\frac{1}{n}\sum_{i = r}^{n} \left(\frac{r - 1}{i - 1}\right)^m$$
And this formula is asymptotically the integral
$$\int_{\frac{r - 1}{n}}^{1} \left(\frac{r - 1}{nx}\right)^m \mathrm{d}x
= \frac{x}{- m + 1} \left.\left(\frac{r-1}{nx}\right)^{m} \right|_{\frac{r-1}{n}}^{1}
= \frac{1}{m - 1} \left( \frac{r - 1}{n} - \left(\frac{r - 1}{n}\right)^m\right)$$
Because $m$ and $r$ are both proportional to $n$, this integral tends
to 0 as $n$ grows to infinity.
Hence it is very unlikely that $u^*$ will get $v^*$.
So the performance of simultaneous stopping rule with $m$ slots is
relatively poor.

The problem of the simultaneous stopping rule with $m$ slots is, 
the threshold that we set is not high enough to filter out 
those applicants who are not sufficiently good. While this is what
the simultaneous stopping rule is good at.

Although this two algorithms does not work that well in general case.
In the next chapter we are going to find some more specific models to
show how powerful this two algorithms could be solving this problem.
